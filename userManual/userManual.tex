\documentclass[11pt,professionalfont]{article}
\usepackage{geometry}
\usepackage{amsmath,amssymb}
\usepackage{colortbl}
\usepackage{graphicx}
\usepackage{graphics}
\usepackage{listings}
\usepackage{ragged2e}
\usepackage{fancyvrb}
\usepackage{relsize}
\usepackage{url}
\usepackage[utf8]{inputenc}
\usepackage[T1]{fontenc}
\usepackage[english]{babel}
\geometry{a4paper}
\usepackage{fancyhdr}
\usepackage[usenames,dvipsnames]{xcolor}

\pagestyle{fancy}
\setlength{\topmargin}{-0.5in}
\setlength{\textheight}{9.5in}

\renewcommand\lstlistingname{\small Listado}
\renewcommand\lstlistlistingname{Listados}
\definecolor{gray97}{gray}{.97}
\renewcommand{\labelitemi}{$\cdot$}

 \lstset{
         basicstyle=\small\rmfamily, % estilo basico
         numbers=left,               % numeros de lineas a la izquierda
         numberstyle=\tiny,          % tamaño de los numeros
         %stepnumber=2,               % espacio entre lineas
         numbersep=5pt,              % separacion entre numeros y texto
         tabsize=2,                  % tamaño del tabulador
         numberfirstline=false,      % indica si se numera la primera linea
         extendedchars=true,         % permite caracteres extendidos 
         breaklines=true,            % separación automática de líneas
         keywordstyle=\color{blue},   % estilo de las palabras reservadas
         commentstyle=\color{BrickRed}, %estilo de comentarios
         language=R,
    	 frame=leftline,             % tipo de marco
 %        keywordstyle=[1]\textbf,    % Stil der Keywords
 %        keywordstyle=[2]\textbf,    %
 %        keywordstyle=[3]\textbf,    %
 %        keywordstyle=[4]\textbf,   \sqrt{\sqrt{}} %
         stringstyle=\color{NavyBlue}\ttfamily, % estilo de cadena
         showspaces=false,           % mostrar espacios en blanco?
         showtabs=false,             % mostrar tabuladores?
         %xleftmargin=17pt,
         %framexleftmargin=17pt,
         framexrightmargin=5pt,      % margen del marco a la derecha
         framexbottommargin=4pt,     % margen abajo
         backgroundcolor=\color{gray97}, % color de fondo
         showstringspaces=false      % no se muestran los espacios entre cadenas      
 }
 \lstloadlanguages{% Check Dokumentation for further languages ...
         %[Visual]Basic
         %Pascal
         %C
         %C++
         %XML
         %HTML
         Java
 }
 
\title{\vspace{-3cm}\begin{center} \line(1,0){370} \\\vspace{0.5cm} \end{center}
User manual: R software for learning Marginal AMP Chain Graphs}


\begin{document}

\maketitle
\vspace{-1.5cm}
\begin{center} \line(1,0){370} \end{center}
\vspace{0.5cm}
\tableofcontents
\begin{center} \line(1,0){370} \end{center}

\section{Introduction}

This R code was used for performing the experiments presented in the
paper. It is aIt is focused on testing the features of the algorithm proposed
for learning MAMP chain graphs (MAMPCG). It is organized as a \textbf{RStudio} 
project although it can be used without this tool. 

\medskip

The experiments presented in the paper are directed towards two main directions:

\begin{itemize}
\item to check the performance of the algorithm when learning MAMP chain 
graphs from databases (section 5.1). The experiments consider three random 
models with \textbf{15} nodes and using different probabilities for introducing 
directed, undirected and bidirected edges. The model named \textbf{artificial1} 
was generated with probabilities $0.2$, $0.2$ and $0.6$ respectively and it 
contains $17$ edges ($5$ directed, $4$ undirected and $8$ bidirected). The 
probabilities used for \textbf{artificial2} were $0.2$, $0.6$ and $0.2$; the 
model contains $32$ edges: $3$ directed, $24$ undirected and $5$ bidirected. 
For \textbf{artificial3} the probabilities were $0.6$, $0.2$ and $0.2$; it 
contains $17$ edges: $7$ directed, $4$ undirected and $6$ bidirected.

\item to compare MAMPCG versus PC algorithms and using databases generated 
from Bayesian networks (section 5.2). The experiments consider the following 
set of networks: \textbf{asia} ($8$ nodes, $8$ directed edges), \textbf{sachs} 
($11$ nodes, $17$ directed edges), \textbf{child} ($20$ nodes,  $25$ directed 
edges), \textbf{insurance} ($27$ nodes, $52$ directed edges), \textbf{mildew} 
($35$ nodes and $46$ directed edges), \textbf{alarm} ($37$ nodes and $46$ 
directed edges) and \textbf{barley} ($48$ nodes and $84$ directed edges).
\end{itemize}

The code offers functions for:

\begin{itemize}
\item random generation of databases from Bayesian networks.
\item random generation of MAMPCG models and databases sampling from them.
\item execution of the learning algorithm, checking learned models against 
the true ones (those where data came from) and storing and showing the results.
\item computing significance tests.
\end{itemize}

The software is configured for using the folders described below:

\begin{itemize}
\item \textbf{ddbb}: it stores the databases used for experiments. The paper describes 
a set of experiments using $30$ databases for each model and  $500$, $1000$, $5000$, 
$10000$  and $50000$ as sample sizes. The folder contains as many sub-folders as models 
(Bayesian networks and MAMPCG). These sub folders contains a new sub folder for each sample 
size. For example, \textbf{./ddbb/asia/} will contain folders named $500$, $1000$, $5000$, 
$10000$ and $50000$. These folders are the final containers of databases. The names 
used for identifying the databases have the following  structure:

\begin{equation*}
modelName - sampleSize - id.db
\end{equation*}

\item [] where $sampleSize$ is one of the sample sizes previously mentioned and  
$id$ a number between $1$ and $30$. As an example, the folder \textbf{./ddbb/asia/50000} 
will contain $3aa0$ files with names from \textbf{alarm-50000-1.db} to \textbf{alarm-50000-30.db}.

\item \textbf{networks}: it stores the definition of the models (Bayesian networks and
MAMPCG models). There are a file for each model. The extension of the files described its
content (\textbf{net} for Bayesian networks \textbf{mampcg} for MAMPCG models).

\item \textbf{results}. This folder will store files storing the results of comparing the
product of the learning algorithm with respect to the true model. Each file contains four
values: precision and recall for links and precision and recall for v-structures. These
results can be used for generating tables, executing significance tests, etc.
\end{itemize}
a


\section{Database generation from Bayesian networks}

The generation of databases from Bayesian network is implemented in 
\textbf{generateDatabasesBN.R}, using some auxiliary functions defined in 
\textbf{auxiliarFunctions.R}. One of the most important functions in this last file 
consists of reading the Bayesian network to sample from:  \textbf{readNetFile}. It 
reads the description of the Bayesian network from a  file with \textbf{net} extension. 
All the networks employed for the experiments were taken from \textbf{bnlearn} Bayesian 
network repository \url{http://www.bnlearn.com/bnrepository/} and stored in a folder named
\textbf{networks}. 

\medskip

The file \textbf{generateDatabasesBN} contains a set of functions:

\begin{itemize}
\item \textbf{generateDatabases.R}. Arguments: Baye\-sian network,  number
of files (databases) to generate, number of samples, path where databases
must be stored and Baye\-sian network file name (in order to give a proper name
to databases). It is the main function devoted to produce the random samples.

\item \textbf{generateDatabasesForNet}. Arguments: Bayesian network file
name, number of files (databases) to generate, number of desired samples, path
where Baye\-sian networks (net files) are stored and path for databases storage.
This function uses the previous one after preparing the required elements:
creates the folder where databases will be stored (if needed) and reads the
Bayesian network file.

\item \textbf{generateDatabasesForNets}: function for generating databases 
for all the Baye\-sian networks stored in a given folder (it uses the previous
one).

\item \textbf{generateDatabasesForNetName}: function for creating the
databases for a Baye\-sian network with a given name. It bases its work on 
producing a call to the function \textbf{generateData\-Bases\-ForNet}. 

\end{itemize}

The file \textbf{testGenerateDatabasesBN.R} includes the sentences for producing 
$30$ variants of each sample size ($500$, $1000$, $5000$, $10000$ and $50000$) 
for \textbf{alarm} network. The code in this file is included below:

\begin{small}
\lstset{linewidth=150mm}
\begin{lstlisting}[]{}
# sets relevant paths to nets folder and databases
path <- "./networks/"
pathddbb <- "./ddbb/"

# generates the databases for all these sample sizes
ns <- c(500,1000,5000,10000,50000)

# sets the number of databases to generate for each dataset
numberFiles <- 30

# sets the name of the net used for generating samples
netname <- "alarm"

# considers each sample size
for(i in 1:length(ns)){
  # selects the corresponding sample size
  nsamples <- ns[i]
  
  # generate the databases
  generateDatabasesForNetName(path, netname, pathddbb, nsamples, numberFiles)
}
\end{lstlisting}
\end{small}

The first sentences define the environment: path to networks definition, 
path for storing databases, set of sample sizes, number of variants per 
sample size and Bayesian network to generate from.

\medskip

After that a loop (line 15 to 21) iterates on the sample sizes and calls the 
function \textbf{generateDatabasesForNetName} passing as arguments the environment 
information. At the end of the loop the databases will be stored into \textbf{ddbb}
folder, using the structure described above.

\section{Random generation of MAMP models and databases}

This functionality is implemented in \textbf{generateMAMPCG.R}. The
file is organized in several functions related to some specific tasks (all
of them focused on generating random models and databases). 

\begin{itemize}
\item Section 1: utility functions for managing edges and nodes, 
getting neighbors, spouses, etc.

  \begin{itemize}
	\item \textbf{getNewFileName(baseFileName, pathNet)}: gets a unique name
	for a new model. This name will contain a prefix given by \textbf{baseFileName} 
	argument (\textbf{artificial} for the experiments) and an \textbf{id} (determined
	in order to avoid repeated names) that will be concatenated to the prefix.

	\item \textbf{createEdge(errorNode, node, edges, leftDeco, rightDeco)}: creates
			a new edge between two nodes with decorations for both sides. 

	\item \textbf{deleteEdge(from, to, edges)}: deletes the edge between the nodes
			$from$ and $to$.

	\item \textbf{deleteEdgeAtRandom(path, edges)}: given a path between nodes
			and the set of edges in the model, it deletes a randomly selected 
			edge in the path. This function is required in order to repair a 
			model in order to guarantee it fulfills the three required conditions 
			for MAMP chain graphs.

	\item \textbf{checkUndirectedEdge(nodeA, nodeB, edges)}: checks if the set
			of edges contains an undirected edge between two nodes.

	\item \textbf{getUndirectedEdges(node, edges)}: gets the set of undirected
			edges in the model containing a given node.

	\item \textbf{getNodesInEdges(node, edge1, edge2)}: gets the nodes involved
			in a given pair of edges, being node a common one for both of them.

	\item \textbf{getOppositeNode(node, edge)}: given an edge this function 
			returns the node involved in the opposite side to node.

	\item \textbf{getSpousesForNode(node, edges)}: gets the set of spouses of
			a given node.

	\item \textbf{getSpouses(edges)}: gets the spouses of each node.

	\item \textbf{getNeighbours(node, edges)}: gets the neighbors of a certain
			node.

	\item \textbf{getNeighboursWithDirections(node, edges}: gets the neighbors
			of node taking into account edges directions. If $A \rightarrow B$
			is in edges, then $B$ is neighbor of $A$ (but $A$ is not neighbor
			of $B$). 

	\item \textbf{getNeighboursWithUndirectedEdges(node,edges)}: gets the neighbors
			of node but taking into account undirected edges only.

	\item \textbf{getPath(from, to=from, visited, edges, neighboursFunction)}: 
			gets the path (if there is such a path) between $from$ and $to$ nodes.
			The function used for deciding the next node to visit is passed as
			the last argument. It is a recursive function.
	\end{itemize}

\item Section 2: MAMPCG functions

  \begin{itemize}
	
	\item \textbf{checkCondition1(edges)}: checks the compliance of the first
			condition of MAMPCG (the graph does not contain undirected cycles).

	\item \textbf{checkCondition2(edges)}: checks second condition
			($G$ has no cycle $V_1 \ldots V_n = V_1$ such that 
			$V_1  \leftrightarrow V_2$ is in $G$ and $V_{i} - V_{i+1}$ is in
			$G$ for all $1 < i < n$).

	\item \textbf{checkCondition3(edges)}: checks third condition (if 
			$V_{1} - V_{2} - V_{3}$ is in $G$ and $spG(V_{2}) \neq \emptyset$, 
			then $V_{1} - V_{3}$ is in $G$ too).

	\item \textbf{checksMAMPCG(edges)}: checks if a model defined by a set of
			edges fulfills the conditions mentioned before. The methods checking
			conditions can modify the set of edges in order to get a valid
			model (adding and removing edges).

	\item \textbf{generateRandomGraph(numberNodes, probDirected, probUndirec\-ted, 					probBidirected)}: base function for the generation. It uses a function of \textbf{bnlearn} 
			package for producing a random graph.  This graph will be used as base for the 
			final model: its arcs are converted to directed, undirected or bidirected at 
			random.

	\item \textbf{generateRandomMAMPCG(numberNodes, probs)}: generates a random
			graph with the desired number of nodes and using the probabilities
			passed as argument for classifying the edges as directed, undirected
			or bidirected. The model is checked in order to guarantee it is a
			valid MAMPCG.
  \end{itemize}

\item Section 3: functions for databases generation

  \begin{itemize} 

	\item \textbf{transformToBayesianNetwork(edges)}: this method
			converts a model into a Bayesian network. This is required in order
			to obtain the samples of the databases. The method produces two
			lists: one with the edges of the Bayesian network obtained from the
			model and another with the set of immoralities. These must be 
			considered afterwards for sampling.

	\item \textbf{createBnlearnNet(edges, check)}: this method uses the set of
			edges created with the previous method. The Bayesian network is
			required for generating the databases.

	\item \textbf{generateDistributionForRootNodesfunction(net)}: generates the
			gaussian distribution for root nodes. The method receives as
			argument the Bayesian network created with the previous function.

	\item \textbf{generateDistributionForNonRootNodes}: generates gaussian
			distributions for nodes with parents. 

	\item \textbf{setParameters(net, params)}: sets the parameters generated for
			root and non root nodes to network.

	\item \textbf{removeEvidentialVariables(sample, evidenceVariables)}: given a
			sample and a list of evidential variables produces a new set of data
			but removing evidential variables.

	\item \textbf{generateSample(model, sampleSize, threshold, cl)}: produces 
			the desired sample using a model passed as first argument. The model
			contains a name (\textbf{artificiali}, being $i$ an unique
			identifier obtained analyzing the folder where models are stored),
			a set of edges, a Bayesian network (derived from the model defined
			by the edges), the complete Bayesian network used for sampling and a 
			set of immoralities produced by the conversion to Bayesian network. The 
			number of samples is given by sampleSize. Threshold is a parameter used in 
			the expression used for the evidence: all the nodes contained in the list 
			of immoralities	 must be included as evidence. Let us assume there are $n$ 
			nodes in the list of immoralities. The evidence expression will have
			the following form:

\begin{multline*}
node_{1} >= - threshold \;\; \& node_{1} <= threshold \;\; \&  \\
\ldots node_{n} >= - threshold \;\; \& node_{n} <= threshold
\end{multline*}

	\item \textbf{storeDatabase(sample, id, numberSamples, path, filename)}:
			stores the samples in a file: the last 4 arguments are used for
			deriving the name of the file to use.

	\item \textbf{storeMode(model, folder}: stores a model in a file in order
			to its posterior use. The model is a list with 5 entries: name,
			set of edges, basic Bayesian network, Bayesian network used for
			sampling (with error nodes, distributions, etc) and list of nodes 
			producing  immoralities. All the models will be stored in files
			with \textbf{mampcg} as extension.

	\item \textbf{retrieveModel(modelname, folder)}: gets a model from a
			file.

	\item \textbf{prepareMAMPCGForSampling(edges, basename, folder}: given a
			set of edges produced by the previous method, it builds a complete
			model ready for databases generation.

	\item \textbf{generateDatabases(model, variants, sampleSizes, threshold,
			pathDb, cluster)}: the model passed as first argument is used for
			producing the desired number of databases for each sample size.
			The databases will be stored in the folder passed as fifth argument.
  \end{itemize}
\end{itemize}

The file \textbf{testGenerateMAMPCG.R} contains an example of generation of
a new model and the corresponding databases. It also shows how can be
retrieved a previous model in order to produce more databases.

\begin{small}
\lstset{linewidth=150mm}
\begin{lstlisting}[]{}
# definition of global parameters: paths, base file name (artificial),
# sample sizes, vector of edges probs for the variants to consider
# (1. 20% for directed edges, 20% undirected, 60% bidirected;
#  2. 20% directed, 60% undirected, 20% bidirected;
#  3. 60% directed, 20% undirected, 20% bidirected)
# number of nets to generate, number of nodes (15), number of databases
# to generate during the sample procedure and seed (NOTE: remove this
# sentence for a real random behaviour; in this cases is fixed just to
# focus the software on the tree models used for the experiments described
# in the paper)
pathDb <- "./ddbb/"
pathNet <- "./networks/"
baseFileName <- "artificial"
sampleSizes <- c(500, 1000, 5000, 10000, 50000)
s
# these are the probs for directed, undirected and bidirected used for
# artificial1, artificial2 and artificial3 respectively
edgesProbs <- list(c(0.2, 0.2, 0.6), c(0.2, 0.6, 0.2), c(0.6, 0.2, 0.2))
numberNets <- 3
numberNodes <- 15
variants <- 30

# This parameter defines the threshold used for the evidence. We have
# used 0.2 for artificial1 and artificial3 and 0.9 for artificial2
threshold <- 0.2

# use parallel execution if possible. The argument defines the number of
# cores to use
cl=makeCluster(2)

# normal procedure
# 1. generates a new graphical model
graphModel <- generateRandomMAMPCG(numberNodes, edgesProbs[[1]])

# 2. prepare the model for sampling. This also requires to get a valid
# name for the model (it will be artificial, bit the id will be assigned
# examining the folder considered for storing the models (pathNet))
completeModel <- prepareMAMPCGForSampling(graphModel, baseFileName, pathNet)

# 3. If needed, store the model for posterior use. The procedure
# will check all the files named artificial and will add a create
# a new one. This method only saves complete models defined by
# edges, a bnet and a set of inmoralities
storeModel(completeModel,pathNet)

# 4. The model can be used to generate the samples
generateDatabases(completeModel, variants, sampleSizes, threshold, pathDb, cl)

# 5. We can retrieve a previously generated model for producing new datasets
completeModel <- retrieveModel("artificial1",pathNet)

# 6. Now it is possible to generateDatasets as before....
# generateDatabases(completeModel, variants, sampleSizes, threshold, pathDb, cl)
\end{lstlisting}
\end{small}

The code begins defining the environment for the generation: path for
databases storage, path to models definition, base name used for the
models (\textbf{artificial}), sample sizes and three  different configurations 
used for producing the models in the paper  (\textbf{artificial1}, 
\textbf{artificial2} and \textbf{artificial3}),  number of nodes,  
number of databases to generate and threshold for evidence expression (lines 1 
to 25). 

\medskip

If possible several cores will be used for speeding up the work (sentence 
giving value to \textbf{cl}; line 29). The first step would be the creation 
of a new graphical model with \textbf{generateRandomMAMPCG}. This model is
completed in order to be used for database generation (\textbf{prepareMAMPCGForSampling}). 
It is convenient to store the model allowing a posterior databases generation, check, etc. 
This can be done using the function \textbf{storeModel}. Anyway, once a model 
is prepared for generating databases the function \textbf{generateDatabases}
can be used. Sometimes it is interesting to use a previous model for producing 
new databases (see steps $5$ and $6$ of code; lines 49 and 52).

\section{Procedure for comparing MAMP versus PC algorithms}

The execution of this comparison requires several conditions:

\begin{itemize}
\item there are a set of Bayesian networks (stored as \textbf{net} format 
files) into a folder named \textbf{networks} (the experiments are based
of learning from \textit{asia}, \textit{sachs}, \textit{child}, \textit{insurance}, \textit{mildew}, \textit{alarm} and \textit{barley}).

\item the set of databases used for the experiments included in the paper 
are stored in a folder named \textbf{ddbb}. Its sub-folders are termed as 
the corresponding Bayesian network. As we have tested several samples sizes, 
each sub folder contains itself several sub folders ($500$, $1000$, $5000$, 
$10000$ and $50000$). These sub folders contain the databases, being $30$ 
variants for each sample size. 
\end{itemize}

The main file for this experiment is \textbf{executeExperiment.R}. It
contains two functions: \textbf{learn} and \textbf{execute}.

\begin{enumerate}
  \item \textbf{learn}, with arguments defining the path to net files, path
  to databases, net name, id of the database to use, number of samples in
  the database, moral (boolean flag stating the way to deal with v-structures),
  pc (boolean flag denoting the algorithm to use; false for MAMPCG algorithm
  and true for PC algorithm), mode (string storing the kind of true model to
  compare with: \textbf{net} for Bayesian networks and \textbf{mampcg} for
  MAMPCG models. This function makes the following operations:

  \begin{itemize}
	\item reads the true model.
	\item reads the database.
	\item builds an object of \textbf{MAMPCGSearch} class. This class contains
          the data and methods needed for the execution of the learning algorithms
		  used in the paper: MAMPCG and PC algorithms. The behaviour of the object
		  is defined through \textbf{pc} and \textbf{mode} flags.
  \end{itemize}
 
  \item \textbf{execute}, with arguments for the path to network file, database,
  net name, database id, number of samples, mode (kind of true model) and debug    
  flags. The flag \textbf{mode} sets up the experiments to perform.  MAMPCG algorithm 
  is required for both experiments and PC algorithm only for the comparison. Therefore 
  the first experiments presented in the paper require applying MAMPCG algorithm on 
  databases derived from MAMPCG models; the second ones will use MAMPCG and PC algorithms 
  on databases coming from Bayesian networks. The operations performed in this method 
  are:

  \begin{itemize}
	\item executes the algorithms.
    \item performs the comparisons between learned and true models.
    \item store the results on files.
  \end{itemize}
\end{enumerate}

The file \textbf{testExecuteComparisonExperiments.R} contains the sentences required for the 
experiments performed on Bayesian networks databases:

\begin{small}
\lstset{linewidth=150mm}
\begin{lstlisting}[]{}
library(foreach)
library(doParallel)

# This sentences configure the environment for performing the
# experiment. Includes the code required for these experiments
source("R/auxiliarFunctions.R")
source("R/utilResults.R")
source("R/MAMPCGSearch.R")
source("R/executeExperiment.R")

#sets the number of digits for the final report
options(digits=2)

# sets the paths to the databases and to the networks. This has to
# be changed when the software is installed on another machine
pathdb <- "./ddbb/"
pathnet <- "./networks/"

# sets the name of the network to work with
netName <- "asia"

# clean ths sink in order to initiate a new trace if required. If this
# is the case, the last of these three lines must be operative removing
# the comment mark
sink()
sink()
traceFileName <- paste0(netName,"-trace")
sink(traceFileName)

# uses parallelism if possible. The next sentence sets the number of
# cores to use (depending on the execution machine)
registerDoParallel(cores=1)
getDoParWorkers()

# set the different sample sizes to consider
samples <- c(500, 1000, 5000, 10000, 50000)

# sets the number of variants for each sample size
repetitions <- 30

cat("Learning process start\n")

# initializes globalResults with an empty list
globalResults <- list()

# sets this var to show the origin of the true model (net or
# edges). Now it is set  to "net" in order to learn a BN. Change
# this value to edges if the objective if to learn (and compare)
# with respect to a mamp model
trueModel <- "net"

# sets debug flag
debug <- FALSE

# considers each sample size
for(i in 1:length(samples)){
  cat("Learning for sample size: ",samples[i],"\n")
  pathdbsample <- ""
  # composes the path where the ddbb is located
  pathdbsample <- paste(pathdb,netName,sep="")
  pathdbsample <- paste(pathdbsample,samples[i],sep="/")
  pathdbsample <- paste(pathdbsample,"/",sep="")
  cat("ddbb path: ",pathdbsample,"\n")
  
  # creates the result matrix for this sample size: it will
  # contain 8 rows (results of mampc and pc) and as many
  # columns as the number of variants for this sample size
  partialResults <- matrix(NA,8,repetitions)
  
  # considers every repetition with a parallel approach
  partialResults <- foreach(j=1:repetitions, .combine='cbind') %dopar% {
    # learn from the ddbb, with ths corresponding sample size and
    # the variant given by j. 
    execute(pathnet, pathdbsample, netName, j, samples[i], trueModel, debug)
  }
  
  # stores the results into globalResults
  colnames(partialResults) <- (c(1:repetitions))
  globalResults[[i]] <- partialResults
}

# generate latex table from data (globalResult) (if required). Only
# one of these sentences must be employed. It is included here in
# order to check the algorithm
generateLatexTableFromData(netName, samples, globalResults)

# generate latex table from files (if required)
#generateLatexTableFromFiles(netName, samples, repetitions)

# gets sure no sink is open
sink()
sink()
\end{lstlisting}
\end{small}
  
The configuration of the execution environment (lines 1 50 53) consists of:

\begin{itemize}
  	\item sets the number of digits to consider for producing the final 
  	output.
  	\item sets the folders containing the databases and the true Bayesian
  	 networks.
  	\item sets the name of the Bayesian network to use (now the script is 
  	prepared for \textbf{asia} network; the assignment of \textbf{netName}
  	variable must be changed to work with another ones).
  	\item prepares the storage of trace messages if required; this set of
  	\textbf{sink} sentences can be commented to avoid the generation of a
  	 trace file.
  	\item sets the parameters for using parallelism taking advantage of the
  	processor cores.
  	\item defines the number of sample sizes to consider for the experiment
  	(giving value to \textbf{samples} variable).
  	\item defines the number of repetitions for each sample size (the value
  	of \textbf{repetitions} is $30$ but it can be changed as well).
  	\item initializes the structure employed for storing the results in
  	memory (\textbf{globalResults}; anyway execution results will be stored
  	in files as well).
  	\item sets a value for \textbf{trueModel} variable. This allow to use 
  	this same set of functions for learning on \textbf{MAMPCG} models (and 
    not Bayesian networks). The learning of Bayesian network requires setting 
  	the value \textbf{``net''} to this parameter.
  	\item defines if debug information will be generated during the execution.
  	This parameter must be set to \textbf{FALSE} except for a detailed trace
  	of the execution.
\end{itemize}

After that the main loop (lines 56 to 80) iterates on the possible values 
of sample sizes. This  loop contains an internal loop (lines 71 to 75) prepared 
for a parallel execution on the databases for a given combination of net and 
sample size. Finally it is possible to gather all the results into a latex table
(line 88). This can be done using the results stored on memory or disk (this 
double mechanism  was implemented just for producing the results when required). 
These lines are commented right now. The script finishes cleaning all the output 
redirection.

\section{Procedure for learning MAMP models}

These experiments try to check the behavior of the algorithm when learning from 
databases coming from MAMPCG models. The execution requires:

\begin{itemize}
\item there are a set of MAMPCG models (stored as rds files; data with 
\textbf{R} binary format and \textbf{mampcg} extension) into \textbf{networks}
folder. The experiments used three artificial models. As it was explained before 
there are functions for producing new random models as well.

\item MAMPCG models can be used for generating random databases with logic
sampling for continuous variables. This was done for the set of networks 
included in the experiments: \textit{artificial1}, \textit{artificial2} 
and \textit{artificial3}. The set of corresponding databases are stored in 
\textbf{ddbb} folder with the same organization explained for databases 
coming from Bayesian networks. 
\end{itemize}

The file \textbf{testExecuteMAMPCGExperiments.R} contains the sentences
required for learning from these databases. Is has a similar structure
to the code used for learning from databases produced from Bayesian networks.
The difference is the value assigned to \textbf{trueModel} variable. This
is enough to change the behavior of the software in order to:

\begin{itemize}
\item reads the true model for a \textbf{R} file with \textbf{mampcg}
extension (instead of using a \textbf{net} file with the definition of
a Bayesian network).

\item executes MAMPCG learning algorithm only (there will not be comparison
respect to PC algorithm).

\item read the databases treating the values as generated from continuous
variables.

\item changes the kind of independence test to perform.

\item changes the way of checking the v-structures of learned versus true
model (using the list of edges for both models).
\end{itemize}

\begin{small}
\lstset{linewidth=150mm}
\begin{lstlisting}[]{}
library(foreach)
library(doParallel)

# This sentences configure the environment for performing the
# experiment
source("R/auxiliarFunctions.R")
source("R/utilResults.R")
source("R/MAMPCGSearch.R")
source("R/executeExperiment.R")

#sets the number of digits for the final report
options(digits=2)

# sets the paths to the databases and to the networks. This has to
# be changed when the software is installed on another machine
pathdb <- "./ddbb/"
pathnet <- "./networks/"

# sets the name of the network to work with
netName <- "artificial1"

# clean ths sink in order to initiate a new trace if required. If this
# is the case, the last of these three lines must be operative removing
# the comment mark
sink()
sink()
traceFileName <- paste0(netName,"-trace")
sink(traceFileName)

# uses parallelism if possible. The next sentence sets the number of
# cores to use (depending on the execution machine)
registerDoParallel(cores=8)
getDoParWorkers()

# set the different sample sizes to consider
samples <- c(500, 1000, 5000, 10000, 50000)

# sets the number of variants for each sample size
repetitions <- 30

# shows the start of the learning process
cat("Learning process start\n")

# initializes globalResults with an empty list
globalResults <- list()

# sets this var to show the origin of the true model (net or
# edges). Now it is set  to "net" in order to learn a BN. Change
# this value to edges if the objective if to learn (and compare)
# with respect to a mamp model
trueModel <- "mampcg"

# sets debug flag
debug <- FALSE

# considers each sample size
for(i in 1:length(samples)){
  cat("Learning for sample size: ",samples[i],"\n")
  pathdbsample <- ""
  # composes the path where the ddbb is located
  pathdbsample <- paste(pathdb,netName,sep="")
  pathdbsample <- paste(pathdbsample,samples[i],sep="/")
  pathdbsample <- paste(pathdbsample,"/",sep="")
  cat("ddbb path: ",pathdbsample,"\n")
  
  # creates the result matrix for this sample size: it will
  # contain 8 rows (results of mampc and pc) and as many
  # columns as the number of variants for this sample size
  partialResults <- matrix(NA,8,repetitions)
  
  # considers every repetition with a parallel approach
  partialResults <- foreach(j=1:repetitions, .combine='cbind') %dopar% {
    # learn from the ddbb, with ths corresponding sample size and
    # the variant given by j. 
    execute(pathnet, pathdbsample, netName, j, samples[i], trueModel, debug)
  }
  
  # stores the results into globalResults
  colnames(partialResults) <- (c(1:repetitions))
  globalResults[[i]] <- partialResults
}

# generate latex table from data (globalResult) (if required). Only
# one of these sentences must be employed. It is included here in
# order to check the algorithm
generateLatexTableFromData(netName, samples, globalResults)

# generate latex table from files (if required)
#generateLatexTableFromFiles(netName, samples, repetitions)

# gets sure no sink is open
sink()
sink()
\end{lstlisting}
\end{small}

\section{Significance test computation}

The file named \textbf{significanceTest.R} contains a single function
for comparing the results of executing MAMPCG and PC algorithms determining
if there is a significant difference between them. This function is based
on the existence of result files previously generated. Its code contains
a main loop for iterating over the sample sizes. Given a concrete sample
size the method gathers the results for all the variants analyzed (all the
databases used for this sample size) and compares the series of values:
for precision, recall, precision for v-structures and recall for v-structures.
The file \textbf{testSignificanceTest.R} shows an example of use for 
\textbf{asia} network:

\begin{small}
\lstset{linewidth=150mm}
\begin{lstlisting}[]{}
# these sentences show how to perform the analysis for a given
# network. The experiments for this network must be already
# available
netName <- "asia"
samples <- c(500, 1000, 5000, 10000, 50000)
repetitions <- 30

# prepares the output to a file
sink()
sink()
sink()

# compose the name of the file to generate with the results
traza <- paste0(netName,"-tests")

# redirect output to the file
sink(traza)

# perform the test
results <- significanceTest(netName, samples, repetitions)
sink()
sink()
\end{lstlisting}
\end{small}

\end{document}
